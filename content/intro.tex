
\chapter*{Vorwort}

\subsection{Regeln}

\DndDropCapLine{O}{akwood Avenue 17} ist ein Solo-Abenteuer, in dem du als Leser*in entscheidest, wie es weitergeht.
Am Ende jedes Abschnitts wirst du vor die Wahl gestellt, wie es weitergehen soll.
Zu jeder Option sind Nummern angegeben, die dir sagen, bei welchem Textblock du weiterlesen musst.
Damit verändert sich die Handlung je nachdem, welche Entscheidungen du triffst!

Hin und wieder können im Verlauf des Abenteuers Gegenstände gefunden werden.
Jedem Gegenstand ist eine Nummer zugeordnet (beispielsweise: ``Du findest einen kleinen Stein \getItem{itemSmallStone}'').
Schreib dir den Namen des Gegenstands mit der Nummer auf.
Einige Gegenstände können bestimmte Entscheidungen ermöglichen oder Ereignisse auslösen.
An dieser Stelle wird auf die Nummer verwiesen.
Wenn du den Gegenstand mit der zugehörigen Nummer besitzt, kann diese Entscheidung gewählt werden.
Die Beschreibungen und Eigenschaften der Gegenstände sind im Appendix angegeben.

Neben Gegenständen kann es auch nötig sein, gewisse Ereignisse zu notieren.
Im späteren Verlauf der Geschichte können sich neue Optionen ergeben, wenn ein Ereignis bereits eingetreten ist.
Beispielsweise kann an einer Stelle stehen ``Markiere Ereignis \getEvent{tutorialMarker}''.
Später ist eine neue Option verfügbar, wenn \getEvent{tutorialMarker} bereits eingetreten ist.
Es reicht, die zugehörigen Buchstaben zu notieren, wenn dazu aufgefordert wird.
Eine kurze Übersicht der möglichen Ereignisse befindet sich im Appendix.

\subsection{Danksagungen}

Es ist immer gut, wenn man weiß, wem man seinen Erfolg zu verdanken hat. Deswegen soll auch dir nicht verschwiegen werden, wer uns geholfen hat.

Wir fangen mit den wichtigsten Leuten an: unseren erschreckend coolen Spendern auf \url{https://patreon.com/tripletwenty}.
Deswegen vielen Dank an Isoboy, Tim, Chris, Matze, Nikolai, foobar, Paul, Frederik, Alexandra, Atlan und Thomas!
Was auch irgendwie komisch ist, weil Thomas ja mitgeschrieben hat, aber Regeln sind Regeln.

Für das Layout wurde das DND5e-LaTeX-Template von rpgtex verwendet (verfügbar unter \url{https://github.com/rpgtex/DND-5e-LaTeX-Template}).
Es bietet ziemlich viele coole Optionen, wenn ihr also gern mit LaTeX arbeitet und Homebrew-Material für eure Rollenspiele schreibt, werft einen Blick darauf!

Zuletzt ein herzliches Dankeschön an die Menschen, die das Abenteuer mehrfach getestet und mit ihrem Feedback geformt haben: %TODO
Danke, dass ihr unsere manchmal seltsamen Formulierungen hinterfragt und so manche Logiklücke gefunden habt.

Und vielen Dank an dich, dass du dir die Mühe gemacht hast das Abenteuer herunterzuladen und zu es spielen willst.
Wir hoffen es bereitet dir einige spaßige Minuten.
Falls du ebenfalls Feedback hast, weil dir etwas besonders gut gefällt oder du ein Problem hast, kannst du gern einen Kommentar unter \url{https://tripletwenty.net/} hinterlassen,
oder eine E-Mail an \textit{dm@tripletwenty.net} schreiben.
Oder du suchst uns auf Twitter (Hinweis: \url{https://twitter.com/tripletwenty_}).
Wir sind sehr, sehr dankbar für jedes Feedback, selbst wenn du nur mitteilen willst, dass du die Geschichte gelesen hast.
Das erwärmt unsere Herzen.


\chapter*{Einleitung}

\subsection{Ein großzügiges Angebot}

\DndDropCapLine{D}{u}, warst zunächst etwas misstrauisch, als man dir die Stelle anbot.
Und ehrlich gesagt bist du es immernoch.
Ein recht einfacher Job als Betreung eines Anwesens in einer kleinen Stadt irgendwo im Nirgendwo.
Was dir daran seltsam vorkam, war das Gehalt, das deutlich höher war, als es für diese Arbeit sein sollte.
Der Vermittler, Herr Pottle, hatte dir erklärt, dass sich in der Stadt niemand gefunden hatte, der die Arbeit übernehmen wollte.
Außerdem ist der Besitzer des Hauses, Scheich Naasif el-Fayad, mehr als wohlhabend, sodass das Gehalt bei den Verhandlungen keine Rolle spielte.
Ehrlich gesagt warst du etwas überrascht, als Herr Pottle das aus deiner Sicht zu hohe Angebot sofort annahm.
Drei Tage und eine lange Zugreise später stehst du nun vor dem Anwesen in der Oakwood Avenue 17...

\paragraph{Der erste Tag}

Der graue, wolkenbedeckte Himmel macht den ersten Eindruck nicht unbedingt besser.
Wie ein bedrohlicher Fels ragt das alte Backsteingebäude aus dem Boden.
Die dunklen Dachziegel sind zum Teil mit Moos überzogen und die hölzernen Dachfiguren sind nur noch entfernt als Adler zu erkennen.
Hinter den Fenstern sind dicke, tiefrote Vorhänge zugezogen.
Nur das oberste Fenster auf der linken Seite des Hauses scheint nicht verdeckt zu sein.

Am eisernen Gartentor ist der abgeblätterte Schriftzug ``Dower'' zu erkennen, vermutlich die früheren Besitzer.
Du kramst in deiner Tasche nach dem schweren Schlüsselbund, das dir Herr Pottle bei deiner Abreise gegeben hat.
