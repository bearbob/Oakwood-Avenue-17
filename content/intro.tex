
\chapter*{Vorwort}

\subsection{Regeln}

\DndDropCapLine{O}{akwood Avenue 17} ist ein Solo-Abenteuer, in dem du als Leser*in entscheidest, wie es weitergeht.
Am Ende jedes Abschnitts wirst du vor die Wahl gestellt, wie es weitergehen soll.
Zu jeder Option sind Nummern angegeben, die dir sagen, bei welchem Textblock du weiterlesen musst.
Damit verändert sich die Handlung je nachdem, welche Entscheidungen du triffst!

Hin und wieder können im Verlauf des Abenteuers Gegenstände gefunden werden. 
Jedem Gegenstand ist eine Nummer zugeordnet (beispielsweise: ``Du findest einen kleinen Stein \getItem{itemSmallStone}''). 
Schreib dir den Namen des Gegenstands mit der Nummer auf. 
Einige Gegenstände können bestimmte Entscheidungen ermöglichen oder Ereignisse auslösen. 
An dieser Stelle wird auf die Nummer verwiesen. 
Wenn du den Gegenstand mit der zugehörigen Nummer besitzt, kann diese Entscheidung gewählt werden. 
Die Beschreibungen und Eigenschaften der Gegenstände sind im Appendix angegeben. 

Neben Gegenständen kann es auch nötig sein, gewisse Ereignisse zu notieren. 
Im späteren Verlauf der Geschichte können sich neue Optionen ergeben, wenn ein Ereignis bereits eingetreten ist.
Beispielsweise kann an einer Stelle stehen ``Markiere Ereignis \getEvent{tutorialMarker}''.
Später ist eine neue Option verfügbar, wenn \getEvent{tutorialMarker} bereits eingetreten ist. 
Es reicht, die zugehörigen Buchstaben zu notieren, wenn dazu aufgefordert wird. 
Eine kurze Übersicht der möglichen Ereignisse befindet sich im Appendix.
%TODO

\chapter*{Einleitung}

\subsection{Vorgeschichte}

\DndDropCapLine{V}{or langer Zeit}, so erzählt man sich, stand hier die Einleitung eines krass epischen Abenteuers.
%TODO

\paragraph{Ein großzügiges Angebot}

Du warst zunächst etwas misstrauisch, als man dir die Stelle anbot.
Und ehrlich gesagt bist du es immernoch.
Ein recht einfacher Job als Betreung eines Anwesens in einer kleinen Stadt irgendwo im Nirgendwo.
Was dir daran seltsam vorkam, war das Gehalt, das deutlich höher war, als es für diese Arbeit sein sollte.
Der Vermittler, Herr Pottle, hatte dir erklärt, dass sich in der Stadt niemand gefunden hatte, der die Arbeit übernehmen wollte.
Außerdem ist der Besitzer des Hauses, Scheich Naasif el-Fayad, mehr als wohlhabend, sodass das Gehalt bei den Verhandlungen keine Rolle spielte.
Ehrlich gesagt warst du etwas überrascht, als Herr Pottle das aus deiner Sicht zu hohe Angebot sofort annahm. 
Drei Tage und eine lange Zugreise später stehst du nun vor dem Anwesen in der Oakwood Avenue 17...
